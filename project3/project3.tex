% Project 3: Research Project
% Author: Rob Kelly
%
% CSE/IT 489/589-06: Introduction to Neural Network Applications
% Spring 2016
% New Mexico Tech
%
%
%

\documentclass[11pt]{cselabheader}

% \usepackage{enumitem}

\fancyhead[R]{Project 3: Neural Networks Research Project}
\title{Neural Network Applications -- Project 3 \\ Neural Networks Research Project}

\begin{document}
\maketitle

\horrule{0.5pt}\\\horrule{2pt}

\section{Introduction}

In this research project, you will work with your group to research a topic of your choice. The topic of your research will involve applying artificial neural networks to solve some real-world problem -- for example, using neural nets for optical character recognition, or a neural net algorithm to forecast a stock-market price. Your research must involve applying the LVQ algorithm and backpropagation-trained ANNs to the problem and analytically comparing the effectiveness of each, but it need not be limited to just those two algorithms.

You'll be working on this project for the rest of the semester. First, you'll need to write a project proposal describing your research topic; this is due \emph{one week from the date this is assigned.} After you've begun work on your topic, you'll need to write a progress report documenting the progress of your project. Finally, you will submit a full written report on your project at the end of the semester, and you'll also be presenting your project to the class.

Your research proposal is due \emph{before class} on \textbf{Thursday, May 31}. If you haven't already, start discussing your topic with your group members now!
\pagebreak

\section{Project Requirement}

\subsection{Research Proposal}
Your written research proposal should include the following:
\begin{enumerate}
  \item Background information on your research topic.
  \item Why you think applying neural networks to this topic could yield interesting results.
  \item How you're going to apply LVQ and backpropagation-trained neural network algorithms to your topic.
  \item A (tentative) schedule for your project.
\end{enumerate}
You should also investigate any published work relevant to your topic conducted by other researchers. Be sure to cite this existing research!

\subsection{Progress Report}
Your progress report should summarize the progress you have made thus far on your project, and should outline how you're going to complete whatever you have left. This should include an updated schedule for your project. If you're running behind schedule, you should revise your project goals to reflect this. Likewise, if your project is on-schedule and you've come up with new ideas to investigate, include these as well!

\subsection{Final Report}
Your final report will include:
\begin{enumerate}
  \item An introduction to your research topic with background information and relevant previous research cited.
  \item A description of your research methodology.
  \item Formal presentation of your results.
  \item The conclusions you have drawn from these results.
\end{enumerate}
This final report must be of \emph{professional quality} -- your work should be publishable in an academic journal!

Additionally, you will be presenting your research project to your instructor and classmates. Your presentation should include visual elements (e.g. a slideshow), and you should aim for a 30-minute time frame. Plan for at least 10 minutes of Q\&A.

\pagebreak

\section{Deadlines \& Submission}
\begin{itemize}
  \item Your \emph{Research Proposal} is due one week from today, on
  \textbf{Thursday, May 31} \emph{before class}. Submit it as a PDF document on
  the course Canvas page. Your lab instructor may have comments or suggestions
  for you about your project -- don't dedicate yourself to anything before your
  instructor approves your project!

  \item Your \emph{Progress Report} will be due \textbf{Thursday, April 21}. Submit it
  as a PDF document on Canvas.

  \item For your \emph{Final Report}, you will present your project in-class on
  \textbf{Thursday, May 5}. Your written report will be due \textbf{Saturday, May
    7}. Submit a PDF of your report, as well as a copy of any presentation
  matericals, on Canvas.
\end{itemize}

\end{document}