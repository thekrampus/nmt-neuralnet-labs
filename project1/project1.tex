% Project 1: Character Recognition
% Author: Rob Kelly
%
% CSE/IT 489/589-06: Introduction to Neural Network Applications
% Spring 2016
% New Mexico Tech
%
%
%

\documentclass[11pt]{cselabheader}

% \usepackage{enumitem}

\fancyhead[R]{Project 1: Character Recognition using Neural Networks}
\title{Neural Network Applications -- Project 1 \\ Character Recognition using Neural Networks}

\begin{document}
\maketitle

\horrule{0.5pt}\\\horrule{2pt}

\section{Introduction}

In this programming project, you will work with your group to design a system that can identify and classify a written character from a bitmap using a neural network algorithm. The character data -- bitmap encodings of five different characters -- will be provided for you, and is available on the course Canvas page.

Your group will write two classification models, each based on a neural network algorithm you've learned in class: Hamming-MAXNET and LVQ. For the former, your implementation will be original work (not using an existing library) written in Python, Java, R, Julia, or MATLAB\footnote{If you wish to use a language or framework not mentioned, ask your lab assistant.}. Be sure to use a language everyone in your group will be more-or-less comfortable with! For the LVQ model, you will use WEKA's implementation, using the Java language.

After evaluating the performance of each of these models in the given task, you'll report your process and results in a project report paper.

Your grade will be based on the \textit{functionality}, \textit{completeness}, and \textit{readability} of your code, as well as the \textit{soundness of your reasoning} in your project report. Some aspects of the project have been left intentionally vague -- use what you've learned in lectures, lab, and in your reading to decide what you will need to focus on.

This project is group work, and your individual grade will be weighted based on your contribution to your group. Make sure you are communicating with the rest of your group members!

\textbf{This project is due on Friday, March 11th, 2016 -- two weeks from the assignment date.} Don't delay, start today!

\pagebreak

\section{Project Requirement}

\subsection{Programming}
\begin{enumerate}
  \item Implement the Hamming-MAXNET algorithm to store the five characters. You'll need to provide:
  \begin{enumerate}[i.]
    \item A function to construct a new network to classify a given set of prototype vectors,
    \item A function to use the constructed network to classify a given input vector, and
    \item A function to evaluate the performance\footnote{Which aspects of your implementation you wish to evaluate is your decision. Bear in mind, you will need useful performance metrics for your project report!} of a constructed network.
  \end{enumerate}

  \item Write a classifier to accomplish the same task using WEKA's implementation of LVQ. You will need to provide the same interface as above, although LVQ is trained, not constructed (effectively, you'll just be writing a wrapper for the library's implementation).
  \begin{enumerate}[i.]
    \item \textit{Since WEKA provides implementations of several LVQ variants, you will need to pick one. You will be graded on this decision -- make sure it's a sound one!}
  \end{enumerate}
\end{enumerate}

\subsection{Project Report}
\begin{enumerate}
  \item For each model -- Hamming-MAXNET and LVQ -- you will briefly describe your implementation. This should include any encoding you apply to the input, as well as any known bugs and defects.

  \item Evaluate each model independently. Your evaluations should include the speed of training, the speed and accuracy of classification, and any other metrics of performance you wish to use.
  \begin{enumerate}[i.]
    \item \textit{For the LVQ model, you must justify your choice of LVQ variant.}
  \end{enumerate}

  \item Evaluate the sensitivity of each model to noise and distortion the input, and to different sizes of input. There are multiple ways to distort the input with random noise -- be sure to describe in detail how you generate ``noisy'' input, why you did it in this way, and what conclusions you can draw from it.

  \item Draw a conclusion: given these results, which algorithm should be used for the task of \textit{optical character recognition (OCR)}?
\end{enumerate}

% \subsection{Presentation}

% You will also report your results in a 15 to 20 minute presentation during class. Make sure you reserve \textit{at least} 5 minutes for Q\&A!

\pagebreak

\section{Submitting}

Your group will submit the following in a \texttt{tar} archive to the course Canvas page:
\begin{itemize}
  \item The complete source code for your implementations of each model.
  \item A README text file, with instructions for building \& running your code.
  \item Your project report document, as a PDF.
  % \item Any visual materials used in your presentation (e.g. slideshow).
\end{itemize}

\textit{Extra work beyond the above specification may be included for extra credit. Extra credit will only be granted to projects which have \textbf{completely fulfilled the base requirements}. All independent research must be properly cited. If you would like to include work for extra credit, you \textbf{must} get pre-approval from your lab assistant.}

\end{document}